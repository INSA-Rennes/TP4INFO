\documentclass[a4paper]{article}

\usepackage[utf8]{inputenc}
\usepackage[french]{babel}
\usepackage{listings}

\begin{document}
\title{DS 2011 Parallélisme}
\author{Damien Crémilleux}
\date{\today}

\maketitle


\section{Question de cours}

\begin{itemize}
\item Tout partage de variable impose une synchronisation :

  Vrai, la synchronisation est nécessaire pour lire et écrire de façon cohérente.

\item Les compteurs de synchronisation sont un mécanisme de syncronisation de bas niveau :

  Faux, il s'agit d'un mécanisme de haut niveau. En effet, le réveil est implicite, on ne l'écrit pas.

\item En Java l'attribut private rend une variable locale à un thread et il faut utiliser shared pour indiquer qu'elle est partagée :

  Faux

\item En Java on choisit quel est le thread que l'on réveille avec la méthode notify(Thread) :
  
  Faux, on ne choisit pas le thread qui sera réveillé.

\item En Java Notify réveille un thread qui reprend instantanément le contrôle de l'objet partagé :
  
  Faux, il faut attendre l'acquisition du verrou (d'où le \textit{while} pour l'attente). C'est donc différent par rapport aux moniteurs de Hoare.

\item Les méthodes \textit{synchronized} s'exécutent en exclusion mutuelle avec toutes les autres méthodes du même objet :

  Faux, les méthodes non-\textit{synchronized} ne sont pas impactées. En outre, les \textit{synchronized} sont réentrants (appel à une méthode \textit{synchronized} dans une méthode déjà \textit{synchronized}).

\item Un programme parallèle va toujours plus vite qu'un programme séquentiel :

  Faux, à cause des synchronizations un programme parallèle peut être plus long qu'un programme séquentiel.

\item Un programme parallèle ne peut s'exécuter que sur une machine parallèle ou multi-coeur :
  
  Faux, il suffit que la machine supporte le multi-threading.

\item La communication par rendez-vous est le moyen de communication le plus rapide dans un programme parallèle exécuté sur un réseau de machines :
  
  Faux, en effet il faut attendre l'émetteur (principe de l'accusé de réception).

\item MPI utilise la communication par rendez-vous
  
  Faux, MPI utilise la communication avec envoi non-bloquant.

\item Les variables globales dans un programme C/MPI sont partagées entre les processus

  Faux, il n'y a pas de variables globales en C/MPI (pas de mémoire partagée non plus).

\item MPI permet le partage de variable

  Faux.

\end{itemize}

\section{Exercice 1 : Synchronisation}

\lstinputlisting[frame=single, language=java, title=\lstname]{Journal.java}

\section{Exercice 2 : MPI}


\end{document}
